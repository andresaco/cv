%!TEX TS-program = xelatex
\documentclass[]{friggeri-cv}

\usepackage{parskip}

\begin{document}
\header{jose andrés}{pizarro}
       {Data engineer | Product Owner | Engineering Manager}


% In the aside, each new line forces a line break
%----------------------------------------------------------------------------------------
%	SIDEBAR SECTION
%----------------------------------------------------------------------------------------

\begin{aside} % In the aside, each new line forces a line break
  \section{bio}
  born on 11/28/1982
  in Cuenca, Spain
  EU-citizen
  \section{contact}
  Isabel Clara Eugenia 4. 
  Portal B. Piso 1C.
  28050 Madrid
  +34 666 325 575
  j.andres.pizarro@gmail.com
  \section{languages}
  {\color{lightgray} $\ast$}{\color{lightgray} $\ast$}{\color{lightgray} $\ast$}{\color{lightgray} $\ast$}{\color{lightgray} $\ast$} español
  {\color{lightgray} $\ast$}{\color{lightgray} $\ast$}{\color{lightgray} $\ast$}{\color{lightgray} $\ast$} english
  \section{programming}
  Java, Python, Shell,
  MySQL, MongoDB, Javascript,
  Eclipse RCP, Java UI Toolkits
  \section{ +technical}
  Windows, UNIX,
  MS Office \& \LaTeX
\end{aside}

%----------------------------------------------------------------------------------------
%	MOTIVATION SECTION
%----------------------------------------------------------------------------------------

\section{motivation}

As {\headingfont head of the advanced analytics product} for BBVA, I work on its strategy to deliver new features that add value to end users, without neglecting the possibility of discovering new features in initiatives that involve both end users and stakeholders. \\
Currently {\headingfont I manage a team of 16 developers} organized in a single Scrum. Although such a large team may sound complex, this responds to my constant willingness to meet the team's requests so that they can work in the best possible way. \\
{\headingfont I consider myself a team person}, supporting my team in everything that is necessary to the best of my ability. \\
{\headingfont All this responsibility comes after more than 10 years working as a developer in the aerospace and banking industries}, developing from near-real time software to data pipeline tools and processes. Not only the technical part is worth mentioning, but also the motivation to communicate, share knowledge and the idea of obtaining continuous feedback from users. \\

%----------------------------------------------------------------------------------------
%	EDUCATION SECTION
%----------------------------------------------------------------------------------------

\section{education}

\begin{entrylist}
%------------------------------------------------
\entry
{2000--2005}
{BSc. + MSc. {\normalfont in Computer Science Engineering}}
{UPM, Madrid, Spain}
{Real-time development, project engineering, software engineering, operating systems}
%------------------------------------------------
\entry
{2017}
{{\normalfont Scrum Fundamentals}}
{scrummanager.net}
{Agile software workflow fundamentals}
%------------------------------------------------
\entry
{2021}
{{\normalfont The Power MBA}}
{thepowermba.com}
{Product development and strategy, management}
%------------------------------------------------
\entry
{2022}
{{\normalfont Engineering Management Workshop}}
{engineeringmanager.academy}
{Engineering Management fundamentals}

\end{entrylist}


%----------------------------------------------------------------------------------------
%	WORK EXPERIENCE SECTION
%----------------------------------------------------------------------------------------
\clearpage
\section{professional experience}

\begin{entrylist}
%------------------------------------------------
\entry
{2007-2018}
{GMV Aerospace and Defence S.A.U.}
{Madrid, Spain}
{\emph{Software Developer} \\
Software engineer leading different software projects involving Java and Python technologies in the aerospace industry, focusing on near real-time performance.
These projects include Desktop applications, backend services and software that requires hardware integration.}
%------------------------------------------------
\entry
{2009-2010}
{SES Astra}
{Betzdorf, Luxembourg}
{\emph{Software Developer} \\
Software engineer in charge of designing and developing a satellite automation tool codenamed SPELL (https://sourceforge.net/projects/spell-sat/).
This automation suite consists on a server implemented in Python an client implemented using eclipse RCP.
Control automation is based on python modules that include a set of domain-specific primitives that allows monitoring and interacting over the
spacecraft status.}
%------------------------------------------------
\entry
{2018-2020}
{DATIO}
{Madrid, Spain}
{\emph{Big Data Developer} \\
Big Data engineer leading the set up, provision and deployment of Python-based solutions for Apache Spark processes.
Core concepts associated to Python deployment, including python project archetypes, libraries and runtime artifacts provision, testing techniques are
generated according to Datio requirements. Additionally, some of my daily tasks include the release of learning resources for Python apprentices.}
%------------------------------------------------
\entry
{2021--}
{DATIO Big Data}
{Madrid, Spain}
{\emph{Product Owner / Engineering Manager} \\
As a Product Owner, I lead the strategy of advanced analytics products based on the Data Platform of BBVA. It includes:
\begin{itemize}
  \item Negotiate with stakeholders the deliverables to be developed for future releases.
  \item Communicate with final users the features delivered, both in presentations and workshops.
  \item User needs processing towards discovering candidate features to be included in future releases.
\end{itemize}
Part of my job also consists on managing a team of 16 individual contributors both backend and frontend developers.
\begin{itemize}
  \item Coordinate the work to be done accross the Sprints, prioritizing the needs and working towards clearing expectations.
  \item Setting the goals and individual expectations of the team members, specially during the one-on-one meetings.
  \item Promoting onboarding initiatives for incoming members.
  \item Coordination with other teams any other initiative whose scope extended the advanced analytics team.
\end{itemize}}
%------------------------------------------------
\end{entrylist}

%----------------------------------------------------------------------------------------
%	EVENTS SECTION
%----------------------------------------------------------------------------------------

\section{events}

\textbf{\headingfont PyCon 2021 speaker} \href{https://www.youtube.com/watch?v=U8SP_Osd8ZY}{Documenta sin ser un escriba} (Document without being a scribe)
 
%----------------------------------------------------------------------------------------

\end{document}